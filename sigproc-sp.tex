% THIS IS SIGPROC-SP.TEX - VERSION 3.1
% WORKS WITH V3.2SP OF ACM_PROC_ARTICLE-SP.CLS
% APRIL 2009
%
% It is an example file showing how to use the 'acm_proc_article-sp.cls' V3.2SP
% LaTeX2e document class file for Conference Proceedings submissions.
% ----------------------------------------------------------------------------------------------------------------
% This .tex file (and associated .cls V3.2SP) *DOES NOT* produce:
%       1) The Permission Statement
%       2) The Conference (location) Info information
%       3) The Copyright Line with ACM data
%       4) Page numbering
% ---------------------------------------------------------------------------------------------------------------
% It is an example which *does* use the .bib file (from which the .bbl file
% is produced).
% REMEMBER HOWEVER: After having produced the .bbl file,
% and prior to final submission,
% you need to 'insert'  your .bbl file into your source .tex file so as to provide
% ONE 'self-contained' source file.
%
% Questions regarding SIGS should be sent to
% Adrienne Griscti ---> griscti@acm.org
%
% Questions/suggestions regarding the guidelines, .tex and .cls files, etc. to
% Gerald Murray ---> murray@hq.acm.org
%
% For tracking purposes - this is V3.1SP - APRIL 2009

\documentclass{acm_proc_article-sp}

\begin{document}

\title{Sping 2018 CS755 Project Report Acoustic Sensing}

\numberofauthors{2}
\author{
\alignauthor
Mingrui Han\\
       \affaddr{George Mason University}\\
       \affaddr{4400 University Drive}\\
       \affaddr{Fairfax, VA}\\
       \email{mhan8@gmu.com}
% 2nd. author
\alignauthor
Joshua Lilly\\
       \affaddr{George Mason University}\\
       \affaddr{4400 University Drive}\\
       \affaddr{Fairfax, VA}\\
       \email{jlilly3@gmu.com}
}

\maketitle
\begin{abstract}
More and more IoT devices emerge into our daily life, aiming to ease
our living style. Those devices such as smart TV, smart fridge
will have more functionality to provide a better user experience.
A traditional controller that uses buttons cannot meet the expectation
of the usage. In this paper, we implement AAmouse \cite{yun:turning}, a system that tracks the device using acoustic sensing. It enables
users to use their mobile devices as the controller to
draw freely in the air. AAmouse adopts inaudible sound wave and Doppler shift effect to track the device movement. Human being cannot
hear the sound wave frequency above 17kHz. However, these frequencies 
are supported in commercial devices. AAmouse takes advantage of this
unused frequencies. Also, AAmouse does not require specialized 
hardware to operate. We implemented AAmouse in two platforms, PC and Android tablet. To refine the tracking result, we applied maximum
ratio combining and signal smoothing filter. Then we conducted the 
evaluation experiment. The actual error is larger than the AAmouse
paper. We also discussed the possible reason of the error. In 
addition, the difference between our implementation and the 
original work is discussed in this work as well.
\end{abstract}

\keywords{AAmouse, acoustic sensing, inaudible sound wave}

\section{Introduction}
In this work, we implemented AAmouse, a system that utilize 
acoustic signal and Doppler shift effect to track the device movement.
AAmouse uses inaudible sound wave that operates above 17kHz.
Researches indicate that human being cannot hear the sound wave above
17kHz. However, current commercial speakers support those frequencies.
The unique property of AAmouse is that it can operates in our daily 
devices and it does not require any specialized hardware. A system
that uses mobile device to track its movement can offer a variety
of usage. User would be able to control the appliance anywhere they
want by carrying their mobile devices and such system can also 
enable user to have a better user experience than the traditional
remote controller. Moreover, mobile devices have IMU sensors built in,
which can also be used to aid the accuracy of the tracking result.

The original AAmouse uses STFT, Doppler shift, maximum ratio 
combining (MRC) Kalman filter, and 
particle filter. Our system is different than the original work with
minor changes. 

Our system sends the sound wave and analyze the received signal using
Doppler shift effect. The frequency shift will give information to 
calculate the velocity. We then used the velocity to calculate the
distance that the device traveled. To refine the tracking result,
we further apply MRC and signal smoothing
filter. 

We implemented the system in PC using matlab and in Android using
Java. Some technical details we are using are different than the 
original paper, but the overall concepts are the same. During the 
experimental evaluation part, we observed that the actual system has
larger error than the original paper.
We provided detailed
reasoning to explain what the potential causes are and what are the
potential solutions.

The rest of this paper is organized as follows. Section
\ref{related_work} introduces some related works in this area.
Section \ref{applications} provides some possible applications of
this work in the future. Section \ref{roles and collaboration}
summarized the contribution of each group member in this work. 
Section \ref{project_overview} provides an overview of the project,
compared to the original work. Section \ref{design_and_implementation}
includes all the technical details of the implementation. We evaluated
the system in section \ref{evaluation}.

\section{Related Work} \label{related_work}
AAmouse is the first tracking system using acoustic signals. It uses
the unused inaudible sound wave that are available in commercial 
devices.
Other
than AAmouse, some other works also utilize acoustic signals. 
CAT \cite{Mao:2016:HAM:2973750.2985617} propose a high precision acoustic tracking using the combination of FMCW with Doppler shift.
It modifies the traditional FMCW which sends a chirp signal and 
applied the modified FMCW to a distributed system. Then it combines 
with Doppler shift to refine the results. The evaluation shows that
this system outperforms AAmouse.
Wang \cite{Wang:2016:DGT:2973750.2973764}
proposed a device free tracking using inaudible sound wave phase 
information. This work differs from previous ones in the fact that it
does not track the movement of a device, instead, it tracks the
movement of the finger. Such device free tracking will have more
applications. For example, it can help to control the smart watch,
and user can wear gloves.
FinerIO 
\cite{Nandakumar:2016:FUA:2858036.2858580} proposed a tracking system
using OFDM and its echo profile. It is also a device free 
tracking system.

In RF-based schemes, ArrayTrack \cite{Xiong:2013:AFI:2482626.2482635}
uses WiFi to realize fine-grained tracking system with a median error of 23 cm using 16 antennas. RF-IDraw 
\cite{Vasisht:2014:RVT:2645884.2645889} 
adopts 8 RFID antennas with different spacing. This system has the
median error of 3.7 cm. WiDraw \cite{Sun:2015:WEH:2789168.2790129}
utilize the angle of arrive(AoA) based on CSI to enable hand-free
drawing. Its median error is 5cm when sing 25 WiFi transmitters.
mTrack \cite{Wei:2015:MHP:2789168.2790113}
has a high accuracy using 60 GHz RF signals.
Tagoram \cite{Yang:2014:TRT:2639108.2639111} 
is also a RFID based tracking system that utilizes
commercial off-the-shelf devices. Its median error is 12 cm.
Compared with RF based tracking systems, audio tracking can achieve
a fine-grained system with high accuracy in mm level.

IMU sensors can also be used for localization tracking. 
For example,
Microsoft X-box Kinect
uses depth sensor and
Nintendo Wii 
uses infrared cameras to track. Both system works for line-of-sight
environment only and they suffer from error accumulation. 
Li \cite{Li:2012:RAI:2370216.2370280}
uses acceleration data from accelerator for localization
tracking. However, it occurs significant error from measurement.
IMU sensor can also be used to improve the tracking result.
Zee \cite{Rai:2012:ZZC:2348543.2348580}
uses such approach to aid the WiFi tracking accuracy. 
Compared
with sensor based tracking systems, audio tracking has a lower 
signal processing cost and it supports condition other than line-
of-sight. 


\section{Applications} \label{applications}
IoT devices becomes more and more popular nowadays. Some appliances
such as smart TV, smart fridge offer more functions than we ever
had before. Merely using a remote controller with buttons cannot meet
the expected user experience offered by those appliances. A mouse has
been on of the best design for controllers. It enables a high level of 
freedom of controlling the devices. There are recently mouse developed
for smart TV to allow users have a better exploration experience. 

A traditional mouse requires a smooth surface to operate, which is 
feasible for a typical PC. However, when it comes to the smart TV
or smart fridge, it is not suitable to have a flat surface near them
and it is not suitable for user to maintain a fixed position when
using the appliance. For example, an user may want to control a TV 
on couch, on desk, or on the dining table. Having a mouse in each
one of the location is not feasible in real world. Our system can be
applied in this scenario by enabling a mouse in the air on the go.
User would be able to control the appliance wherever they want with
rich control options.



\section{Roles and collaboration} \label{roles and collaboration}
We discussed the project implementation details together, including
what we should implement, what the time line would be, what the issues
are, and what we should do to solve them. For each technical concept
we need, such as MRC, Kalman filter, STFT, Doppler Shift, and the
particle filter, we discussed the implementation procedure, 
potential issues, and the possible solutions.

Mingrui Han: Implement the system in Android tablet. Review and discuss matlab code. Implement MRC, distance calculation, following
the matlab code. 
Research and implement FFT, signal filter, and other third party 
tools which java does not have.  

Joshua Lilly:

\section{Project Overview} \label{project_overview}
We implement a modified version of AAmouse in two platforms, PC
(matlab) and Android tablet. Our project is different than the 
original work in some way but it has the same concept as AAmouse.
The key techniques used in AAmouse are, Doppler shift, distance
calculation,  maximum ratio combining (MRC), Kalman filter,
calibration phase, and particle filter. As we implement the work,
we found that the paper omits some detail for implementation. 
Compared with AAmouse, we implemented a modified version of the 
system. Our system uses Doppler shift, distance calculation, 
MRC, and signal smoothing filter. Detail will be given in the later 
sections. 

\subsection{Original Work}
The key idea of the original work is using Doppler shift as an
estimate of the device velocity. The Doppler shift is a well known
effect where the frequency of a signal changes as a sender or a 
receiver moves. In our case, the sender is fixed, so we only 
consider the movement of the receiver. Doppler shift is observed
in both RF wave and acoustic wave. Because of its speed propagation
speed and narrower bandwidth, 
acoustic signal can achieve a higher accuracy than RF signal.
The detailed equation is shown below.

\begin{equation}
v= \frac {F^s}{F} \times C
\end{equation}

\begin{math} F \end{math} denotes the original frequency of the
signal. \begin{math} F^s \end{math} is the frequency shift of the
signal. \begin{math} C \end{math} is the signal speed, which is the
sound wave speed in this case. After the calculation of the velocity
\begin{math} v \end{math}, AAmouse integrates it 
with time to obtain the distance of the device movement.

\begin{equation}
D = D + v \times t
\end{equation}

In 1D tracking, \begin{math} D \end{math} is the result. In 2D
tracking, \begin{math} D \end{math} is the distance for each
speaker. We have 2 distances of the device to each speaker and
we have the distance between the two speakers. We will be able to
track the device movement using basic triangle properties. 

To improve the accuracy, AAmouse uses maximum ratio combining (MRC). 
Measuring from a single frequency may not be reliable. Thus, AAmouse
sends the signal in 10 frequencies, with a guarding frequency of 
200Hz. Then it applied MRC to average the received signal weighted
by the inverse of the noise variance. After performing MRC,
AAmouse adopts Kalman filter to smooth the estimation with noise
covariance and measurement noise covarince both to be 0.00001.

To find the distance between the speakers, AAmouse introduced a 
Doppler shift based calibration process. 
The TV emits the inaudible sound and user scans the TV with his hands
holding the device. The users start from the left end of the TV to the
right end of the TV.
It finds the points where the 
value of Doppler shift change from positive to negative. Those points
indicate the time that uses spent to move the device between the
speakers. 

AAmouse finds the initial device location by applying a particle 
filter. It generates many particles uniformly distributed in the 
area. Each particle indicates a possible initial location of the
device. In the next Doppler shift interval, the system checks the 
device movement of each particle. If the movement is not feasible,
such particle will be filtered out.

\subsection{Project Summary}
Summarize  what  you  have  achieved  in  your  project,  espe-
cially compared to the original work.  What parts were and
were  not  implemented,  performance  gap  from  the  original
work, etc.

\section{Design and Implementation} \label{design_and_implementation}
This should be the main part of the report.  It should in-
clude below subsections.  Feel free to add (do not remove)
subsections and reorder them.

\subsection{Setup}
Scenarios  and  setup  of  your  implementation,  including  lo-
cation,  hardware,  software,  and so on.  If possible,  include
pictures and figures to be illustrative.  Compare your setup
to the original paper, and explain why you did so

\subsection{Technical Details}
All  the  technical  details  related  to  your  project  should  be
placed  here  including,  but  not  limited  to,  techniques  and
methodologies involved, theories behind thee, how they were
implemented, what was the difference from the originial pa-
per,  what  was  not  mentioned  in  the  paper  and  how  you
overcame it, and so on.  Be specific.

\subsection{Lessons learned}
What are the knowledge obtained from this project? Clearly
state what you have learned from this project, beyond what
is on the original paper. This includes practical issues you’ve
faced, how you were able to troubleshoot them.  Even if you
could not fix it, describe what you have tried, why you were
unable to solve it, and your hypothesis on what would be the
cause.  If you were not able to implement the entire system
in the originial paper, explain why.

\section{EVALUATION} \label{evaluation}
Performance evaluation and potential improvement.  Should
include below subsections.  Feel free to add (do not remove)
subsections and reorder them.
%\end{document}  % This is where a 'short' article might terminate

%ACKNOWLEDGMENTS are optional
\subsection{Performance and Analysis}
Be illustrative with graphs and figures.  However, don’t just
simply enlist the results, but explain them and provide anal-
ysis/insights obtained from the implementation and exper-
iments.  For example, why does your project performs bet-
ter/worse than what is reported in the original paper?  Why
does it perform differenty under various scenarios?  Any un-
expected results are found, and why

\subsection{Potential Improvements}
From  your  experiment  experience,  what  do  you  think  can
be done to improve the performance.  Please justify

%
% The following two commands are all you need in the
% initial runs of your .tex file to
% produce the bibliography for the citations in your paper.
\bibliographystyle{abbrv}
\bibliography{sigproc}  % sigproc.bib is the name of the Bibliography in this case
% You must have a proper ".bib" file
%  and remember to run:
% latex bibtex latex latex
% to resolve all references
%
% ACM needs 'a single self-contained file'!
%
\end{document}