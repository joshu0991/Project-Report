% THIS IS SIGPROC-SP.TEX - VERSION 3.1
% WORKS WITH V3.2SP OF ACM_PROC_ARTICLE-SP.CLS
% APRIL 2009
%
% It is an example file showing how to use the 'acm_proc_article-sp.cls' V3.2SP
% LaTeX2e document class file for Conference Proceedings submissions.
% ----------------------------------------------------------------------------------------------------------------
% This .tex file (and associated .cls V3.2SP) *DOES NOT* produce:
%       1) The Permission Statement
%       2) The Conference (location) Info information
%       3) The Copyright Line with ACM data
%       4) Page numbering
% ---------------------------------------------------------------------------------------------------------------
% It is an example which *does* use the .bib file (from which the .bbl file
% is produced).
% REMEMBER HOWEVER: After having produced the .bbl file,
% and prior to final submission,
% you need to 'insert'  your .bbl file into your source .tex file so as to provide
% ONE 'self-contained' source file.
%
% Questions regarding SIGS should be sent to
% Adrienne Griscti ---> griscti@acm.org
%
% Questions/suggestions regarding the guidelines, .tex and .cls files, etc. to
% Gerald Murray ---> murray@hq.acm.org
%
% For tracking purposes - this is V3.1SP - APRIL 2009

\documentclass{acm_proc_article-sp}

\begin{document}

\title{Sping 2018 CS755 Project Report Acoustic Sensing}

%
% You need the command \numberofauthors to handle the 'placement
% and alignment' of the authors beneath the title.
%
% For aesthetic reasons, we recommend 'three authors at a time'
% i.e. three 'name/affiliation blocks' be placed beneath the title.
%
% NOTE: You are NOT restricted in how many 'rows' of
% "name/affiliations" may appear. We just ask that you restrict
% the number of 'columns' to three.
%
% Because of the available 'opening page real-estate'
% we ask you to refrain from putting more than six authors
% (two rows with three columns) beneath the article title.
% More than six makes the first-page appear very cluttered indeed.
%
% Use the \alignauthor commands to handle the names
% and affiliations for an 'aesthetic maximum' of six authors.
% Add names, affiliations, addresses for
% the seventh etc. author(s) as the argument for the
% \additionalauthors command.
% These 'additional authors' will be output/set for you
% without further effort on your part as the last section in
% the body of your article BEFORE References or any Appendices.

\numberofauthors{2} %  in this sample file, there are a *total*
% of EIGHT authors. SIX appear on the 'first-page' (for formatting
% reasons) and the remaining two appear in the \additionalauthors section.
%
\author{
% You can go ahead and credit any number of authors here,
% e.g. one 'row of three' or two rows (consisting of one row of three
% and a second row of one, two or three).
%
% The command \alignauthor (no curly braces needed) should
% precede each author name, affiliation/snail-mail address and
% e-mail address. Additionally, tag each line of
% affiliation/address with \affaddr, and tag the
% e-mail address with \email.
%
% 1st. author
\alignauthor
Mingrui Han\\
       \affaddr{George Mason University}\\
       \affaddr{4400 University Drive}\\
       \affaddr{Fairfax, VA}\\
       \email{mhan8@gmu.com}
% 2nd. author
\alignauthor
Joshua Lilly\\
       \affaddr{George Mason University}\\
       \affaddr{4400 University Drive}\\
       \affaddr{Fairfax, VA}\\
       \email{jlilly3@gmu.com}
}
% There's nothing stopping you putting the seventh, eighth, etc.
% author on the opening page (as the 'third row') but we ask,
% for aesthetic reasons that you place these 'additional authors'
% in the \additional authors block, viz.

% Just remember to make sure that the TOTAL number of authors
% is the number that will appear on the first page PLUS the
% number that will appear in the \additionalauthors section.

\maketitle
\begin{abstract}
Abstract should be one paragraph summary of the introduc-
tion.  Be sure to highlight your achievements.
\end{abstract}



\keywords{ACM proceedings, \LaTeX, text tagging} % NOT required for Proceedings

\section{Introduction}
This section should summarize the entire paper.

\section{RELATED WORK}
State of the art techniques in literature, and how they are
related to the work you’ve done.  This should not exceed half
page

\section{APPLICATIONS}
Discuss how your technique can be used in real-life applica-
tions.  Be sure to put your thoughts beyond what is written
in the original paper.  This should not exceed half pag

\section{ROLES AND COLLABORATION}
Only for teams.  Who did what and how did you collaborate.
Be VERY specific.

\section{PROJECT OVERVIEW}
This section should provide an overview of the project, espe-
ically compared to the original work.  Please include below
subsections.   You  are  free  to  add  or  reorder  sections,  but
should not remove any.

\subsection{Original Work}
Briefly  summarize  the  original  work.   You  may  use  your
critique  but  please  adjust  to  fit  into  the  project  context.
Should not exceed half a page

\subsection{Project Summary}
Summarize  what  you  have  achieved  in  your  project,  espe-
cially compared to the original work.  What parts were and
were  not  implemented,  performance  gap  from  the  original
work, etc.

\section{DESIGN AND IMPLEMENTATION}
This should be the main part of the report.  It should in-
clude below subsections.  Feel free to add (do not remove)
subsections and reorder them.

\subsection{setup}
Scenarios  and  setup  of  your  implementation,  including  lo-
cation,  hardware,  software,  and so on.  If possible,  include
pictures and figures to be illustrative.  Compare your setup
to the original paper, and explain why you did so

\subsection{Technical Details}
All  the  technical  details  related  to  your  project  should  be
placed  here  including,  but  not  limited  to,  techniques  and
methodologies involved, theories behind thee, how they were
implemented, what was the difference from the originial pa-
per,  what  was  not  mentioned  in  the  paper  and  how  you
overcame it, and so on.  Be specific.

\subsection{Lessons learned}
What are the knowledge obtained from this project? Clearly
state what you have learned from this project, beyond what
is on the original paper. This includes practical issues you’ve
faced, how you were able to troubleshoot them.  Even if you
could not fix it, describe what you have tried, why you were
unable to solve it, and your hypothesis on what would be the
cause.  If you were not able to implement the entire system
in the originial paper, explain why.

\section{EVALUATION}
Performance evaluation and potential improvement.  Should
include below subsections.  Feel free to add (do not remove)
subsections and reorder them.
%\end{document}  % This is where a 'short' article might terminate

%ACKNOWLEDGMENTS are optional
\subsection{Performance and Analysis}
Be illustrative with graphs and figures.  However, don’t just
simply enlist the results, but explain them and provide anal-
ysis/insights obtained from the implementation and exper-
iments.  For example, why does your project performs bet-
ter/worse than what is reported in the original paper?  Why
does it perform differenty under various scenarios?  Any un-
expected results are found, and why

\subsection{Potential Improvements}
From  your  experiment  experience,  what  do  you  think  can
be done to improve the performance.  Please justify

%
% The following two commands are all you need in the
% initial runs of your .tex file to
% produce the bibliography for the citations in your paper.
\bibliographystyle{abbrv}
\bibliography{sigproc}  % sigproc.bib is the name of the Bibliography in this case
% You must have a proper ".bib" file
%  and remember to run:
% latex bibtex latex latex
% to resolve all references
%
% ACM needs 'a single self-contained file'!
%
\end{document}